%
% $Id: poisson.tex,v 1.11 2012/02/29 12:06:46 patrick Exp $
%
% Copyright (c) 2000-2011 Patrick Guio <patrick.guio@gmail.com>
%
% All Rights Reserved.
%
% This program is free software; you can redistribute it and/or modify it
% under the terms of the GNU General Public License as published by the
% Free Software Foundation; either version 2.  of the License, or (at your
% option) any later version.
%
% This program is distributed in the hope that it will be useful, but
% WITHOUT ANY WARRANTY; without even the implied warranty of
% MERCHANTABILITY or FITNESS FOR A PARTICULAR PURPOSE.  See the GNU General
% Public License for more details.
%

\documentclass[10pt,a4paper]{article}
\usepackage{times}
\usepackage{fullpage}
\usepackage{amsmath}
\usepackage[round]{natbib}
\usepackage[latin1,utf8]{inputenc}
\usepackage{graphicx}
\usepackage{fancybox}
\usepackage{tabularx}
\usepackage{needspace}
\usepackage{esdiff}
\usepackage{physics}
\usepackage{xspace}
\usepackage{html}
\usepackage{version}


\parindent 0cm

\def\Eq#1{Eq.~(#1)}

\newcommand*{\matlab}{\textsf{Matlab}\xspace}
\newcommand*{\mgfas}{\textsf{Mgfas}\xspace}
\newcommand*{\mudpack}{\textsf{Mudpack}\xspace}
\newcommand*{\mudfas}{\textsf{Mudfas}\xspace}


\renewcommand{\floatpagefraction}{.9}
\renewcommand{\textfraction}{.1}
\renewcommand{\topfraction}{.8}
\renewcommand{\bottomfraction}{.8}

%\def\tablesize{\footnotesize}
\def\tablesize{\small}

\bibliographystyle{plainnat}

\title{\mudfas: a nonlinear multigrid Poisson equation
solver\\version~\version}
\author{Patrick Guio}
\date{\normalsize$ $Date: 2012/02/29 12:06:46 $ $,~ $ $Revision: 1.11 $ $}

\begin{document}

\maketitle

\tableofcontents

\section{Nonlinear Poisson equation}
We want to solve the Poisson equation on a rectangular domain and with
Boltzmann distributed electrons, i.e.\ in a quasi-static approximation.
This is, in dimensionless variables, a nonlinear elliptic partial
differential equation defined on the rectangular \emph{unit} domain $\Omega$
with condition specified at the boundary $\partial\Omega$. The boundary
condition can be categorised  in three classes. When the potential $\phi$ is
specified at the boundary (insulated boundary), it is a \emph{Dirichlet}
problem
\begin{align}
\left\{\begin{array}{lll}\displaystyle
\grad^2\phi(\vec{r})-\exp\frac{\phi(\vec{r})}{T_e(\vec{r})}&=
-\rho(\vec{r})&,\qquad\forall\vec{r}\in\Omega\\
\phi(\vec{r})&=f(\vec{r})&,\qquad\forall\vec{r}\in\partial\Omega
\end{array}\right.
\label{eq:dirichlet} 
\end{align} 
When the normal derivative of the potential $\phi$ is  specified at the
boundary (conductive boundary), it is a \emph{Neumann} problem

\begin{align}
\left\{\begin{array}{lll}\displaystyle
\grad^2\phi(\vec{r})-\exp\frac{\phi(\vec{r})}{T_e(\vec{r})}&=
-\rho(\vec{r})&,\qquad\forall\vec{r}\in\Omega\\
\frac{\partial\phi(\vec{r})}{\partial{n}}&=f(\vec{r})&,
\qquad\forall\vec{r}\in\partial\Omega 
\end{array}\right.
\label{eq:neumann} 
\end{align}
The Neumann problem can be included in a somewhat wider category
of problem: problems with mixed derivative boundary condition.

The third class of boundary condition \emph{requires} rectangular domains and 
is the periodic boundary condition.

The electron and ion density, $n_e$ and  $n_i$ are defined respectively 
by
\begin{align}
n_i&=\int_\Omega\rho(\vec{r})\dint{\vec{r}}\\
n_e&=\int_\Omega\exp\frac{\phi(\vec{r})}{T_e(\vec{r})}\dint{\vec{r}}
\end{align}
and we would like to insure to a good approximation the quasi-neutrality, i.e.\
$n_e\simeq n_i$.

\section{Linearisation}
With the assumption that $\phi/T_e\ll 1$, this nonlinear Poisson equation can
be linearised and leads to the following linear elliptic differential equation 
\begin{align}
\grad^2\phi(\vec{r})-1-\frac{\phi(\vec{r})}{T_e(\vec{r})}= -\rho(\vec{r})
\label{eq:lpoisson}
\end{align}
This equation can be solved analytically for harmonic plane waves.

\subsection{Harmonic plane waves solution}

In the unit domain $\Omega$, let us consider the periodic ion charge
distribution $\rho(\vec{r})=1+\rho_0\cos(\vec{k}\cdot\vec{r})$, where
$\vec{k}$ is the wave vector of the plane wave, and look for a plane wave
solution of the form $\phi(\vec{r})=\phi_0\cos(\vec{k}\cdot\vec{r})$.
Replacing $\rho$ and $\phi$ into \Eq{\ref{eq:lpoisson}} leads to 

\begin{align}
-k^2\phi_0\exp(i\vec{k}\cdot\vec{r})-1-
\frac{\phi_0}{T_e}\exp(i\vec{k}\cdot\vec{r})
=-1-\rho_0\exp(i\vec{k}\cdot\vec{r})
\end{align}
Assuming in addition that $\vec{k}/2\pi$ is a vector with integer
components, insures quasi neutrality $n_e=n_i$.
The harmonic plane wave solution $\phi$ is then written
\begin{align}
\phi(\vec{r})=\frac{\rho_0}{k^2+1/T_e}\cos(\vec{k}\cdot\vec{r})
\label{eq:lin}
\end{align}

\section{Nonlinear Poisson solvers}

We performed a set of numerical experiments in both 2D and 3D with two
solvers referred hereafter \mgfas and \mudfas. Our aim is to compare their
respective accuracy.

\subsection{\mgfas}
\mgfas denotes the original 2D nonlinear Poisson solver developed by
Steinar Børve. This solver implements a two-grid algorithm adapted from the 
routine \texttt{mgfas}, a
full multigrid algorithm for FAS (Full Approximate Storage) solution of
nonlinear elliptic equation \citep{press:1992}. \mgfas denotes as well the
extension to 3D developed by Lars Daldorff. The 2D version of \mgfas
allows $T_e$ to be any arbitrary function while the 3D version is currently
running with constant $T_e$. 
In both 2D and 3D versions, the boundary conditions are ``hard-coded''.

The parameter configuration in \mgfas has not been modified compared to the
original setting, i.e.\ $\texttt{maxit}\!=\!10000$ and
$\texttt{minerr}\!=\!5.0\mbox{e}^{-4}$.


\subsection{\mudfas}
\mudfas denotes the nonlinear 2D/3D Poisson solver I have first prototyped 
with \matlab and then have written a production version in C++
using the template class library for scientific computing
Blitz++\footnote{http://oonumerics.org/blitz}. It implements a nonlinear
multigrid method that handles, in a unified way, any combination of
periodic, specified (Dirichlet), or mixed derivative boundary conditions
inspired in its principle by the Fortran 77/90 package
\mudpack\footnote{http://www.scd.ucar.edu/css/software/mudpack} for
solving linear elliptic partial differential equations \citep{adams:1989}.  
The basic algorithm is a nonlinear multigrid with a recursive formulation
\citep{wesseling:1991}. 
The boundary conditions are fully steerable by parameter parsing.

\mudfas consists of two schedules. The first schedule, performed
systematically, is a \emph{full multigrid schedule} (FMG) or nested iterations,
with an adaptive strategy which allows to skip coarse-grid correction (the
recursion in the multigrid algorithm) if the specified accuracy is already
reached on that grid \citep{wesseling:1991}. The FMG might be followed 
by maximum $n$ iterations of the \emph{adaptive multigrid schedule} in order
to reach a higher accuracy if required. 

The relaxation algorithm used is a Gauss-Seidel-Newton \citep{press:1992}.
In both 2D and 3D, $T_e$ can be
any arbitrary functions.

The parameter configuration in \mudfas for the numerical experiments is
$\texttt{maxcy}\!=\!2$, $\texttt{kcycle}\!=\!2$ (W-cycle) and 
$\texttt{delta}\!=\!0.5$ (adaptive parameter), 
$\texttt{tolmax}\!=\!5.0\mbox{e}^{-4}$,
$\texttt{iprer}\!=\!2$ (number of pre-relaxation sweeps),
$\texttt{ipost}\!=\!1$ (number of post-relaxation sweeps) and
$\texttt{intpol}\!=\!1$ (multilinear prolongation or interpolation).

\subsection{Boundary conditions}

In order to compare the two solvers, the boundary conditions of \mudfas
have been set to the same type as \mgfas.  
In 2D, these are Dirichlet condition with $\phi=0$ on the edges $x=a$
and $x=b$ and Neumann condition with $\partial\phi/\partial n=0$ on the
edges $y=c$ and $y=d$. 
In 3D, these are Dirichlet with $\phi=0$ on the edges $x=a$, $x=b$, $z=e$ 
and $z=f$ and Neumann condition with $\partial\phi/\partial n=0$ on the edge
$y=c$ and $y=d$. 

\subsection{Accuracy measurement}

In order to evaluate the accuracy of the solution, the defect (the opposite of 
the residual) of the solution is calculated on the finest grid. 
The defect $d_h$ of the solution on the grid with mesh size $h$ is
\begin{align}
d_h(\phi)=\grad_h^2\phi-\exp\left(\frac{\phi}{T_e}\right)+\rho
\end{align}
where $\grad_h^2$ is the  discrete Laplacian operator written in 2D
\begin{align}
\grad_h^2\phi_{i,j}=
\frac{\phi_{i-1,j}-2\phi_{i,j}+\phi_{i+1,j}}{h_i^2}+
\frac{\phi_{i,j-1}-2\phi_{i,j}+\phi_{i,j+1}}{h_j^2}
\end{align}
and in a similar way in 3D.

\section{Numerical experiments}

When not specified, the 2D numerical experiments are performed on a grid of
\input{nxny.dat} corresponding to the domain \input{omega2d.dat} and the 3D
numerical experiments on a grid of \input{nxnynz.dat} corresponding to the
domain \input{omega3d.dat}
The temperature $T_e$ is constant over $\Omega$ and equal to 2 when not
specified.

\subsection{Harmonic plane wave}

\begin{figure}
\centerline{\includegraphics[width=\columnwidth]{test1_2_2d}}
\centerline{\includegraphics[width=\columnwidth]{test1_2_3d}}
\caption{\emph{Upper} row, the defect of the 2D
potential, the parameters are \protect\input{partest1_2d.dat}. \emph{Lower}
row, the defect of a cut into the $xy$-plane
of the 3D potential, the parameters are \protect\input{partest1_3d.dat} }
\label{fig:1_2_23d}
\end{figure}

\begin{table}
\centerline{
\tablesize \begin{tabular}{lcc}
\hline\hline & $m$ & $\sigma$\\ \hline
\csname @@input\endcsname test1_2d.dat
\hline\hline
\end{tabular}
\hspace*{0.5cm}
\begin{tabular}{lcc}
\hline\hline & $m$ & $\sigma$\\ \hline
\csname @@input\endcsname test1_3d.dat
\hline\hline
\end{tabular} 
}
\caption{Mean value and standard deviation of the defect of the
2D potential (\emph{left} table) shown in the \emph{upper} row in 
Figure~\ref{fig:1_2_23d},
and of the 3D potential (\emph{right} table). A cut into the $xy$-plane of the
3D potential is shown in the \emph{lower} row in Figure~\ref{fig:1_2_23d} }
\label{table:test1_23d}
\end{table}

Figure~\ref{fig:1_2_23d} shows the defect of the 
potential given by the linear approximation \Eq{\ref{eq:lin}}, and 
calculated by both solvers in 2D and 3D respectively. 
Table~\ref{table:test1_23d} presents the mean value and standard
deviation of the defect over the domain $\Omega$. 

Note that the accuracy achieved by \mudfas is reached with just running the
FMG schedule, i.e.\ no extra adaptive multigrid schedule is needed in order
to achieve the prescribed accuracy.

\subsection{Wave number influence}

Let us know consider harmonic plane waves along the $y$-axis with different
wave number $k_y$.  For our numerical experiment, we have considered six
different wave numbers $k_y=2\pi$, $4\pi$, $6\pi$, $8\pi$, $10\pi$ and
$12\pi$.  

\begin{table}
\centerline{\tablesize \begin{tabular}{r|rr|rr|rr}
\hline\hline
\multicolumn{1}{c}{}&
\multicolumn{2}{|c}{$\|d_h(\phi_\mathrm{lin})\|$}&
\multicolumn{2}{|c}{$\|d_h(\phi_\mathrm{mgfas})\|$}&
\multicolumn{2}{|c}{$\|d_h(\phi_\mathrm{mudfas})\|$}\\
$k_y$ & $m$ & $\sigma$ & $m$ & $\sigma$ & $m$ & $\sigma$\\
\hline
\csname @@input\endcsname test3_2d.dat
\hline\hline
\end{tabular} }
\vspace*{0.1cm}
\centerline{\tablesize \begin{tabular}{r|rr|rr|rr}
\hline\hline
\multicolumn{1}{c}{}&
\multicolumn{2}{|c}{$\|d_h(\phi_\mathrm{lin})\|$}&
\multicolumn{2}{|c}{$\|d_h(\phi_\mathrm{mgfas})\|$}&
\multicolumn{2}{|c}{$\|d_h(\phi_\mathrm{mudfas})\|$}\\
$k_y$ & $m$ & $\sigma$ & $m$ & $\sigma$ & $m$ & $\sigma$\\
\hline
\csname @@input\endcsname test3_3d.dat
\hline\hline
\end{tabular} }
\caption{Mean value and standard deviation of
the defect of the 2D potential (\emph{upper} table) and the 3D potential
(\emph{lower} table) for the different wave numbers}
\label{table:test3_23d}
\end{table}

Table~\ref{table:test3_23d} presents the mean value  and standard deviation
of the defect for the 2D and 3D potential respectively.  Note again the
higher accuracy achieved by \mudfas with just one FMG. 

There is no tremendous influence of the wave number on the accuracy  even
though the slight deterioration of accuracy at large wave number (short
wave) for \mudfas can be interpreted in the following way.  There are three
grids in both the 2D and 3D \mudfas experiment, and the number of grids
improves the coarse-grid correction, i.e.\ improves the correction on waves
with small wave numbers. At the same time waves with large wave numbers do
not benefit from the coarse-grid correction since their accuracy is achieved
on the fine-grid. Therefore it is expectable to achieve higher accuracy
for short wave number for a given number of grids.

\subsection{Grid number influence}

\mudfas has a scheme which optimises the number of subgrids as a function
of the number of points chosen for the grid. We have performed numerical
experiments in both 2D and 3D with an increasing number of subgrids and with
two waves with wave number \input{partest4_2d.dat} and
\input{partest5_2d.dat} respectively.

\begin{table}
\centerline{\tablesize \begin{tabular}{r|rr|rr|rr}
\hline\hline
\multicolumn{1}{c}{}&
\multicolumn{2}{|c}{$\|d_h(\phi_\mathrm{lin})\|$}&
\multicolumn{2}{|c}{$\|d_h(\phi_\mathrm{mgfas})\|$}&
\multicolumn{2}{|c}{$\|d_h(\phi_\mathrm{mudfas})\|$}\\
grid size $(k)$ & $m$ & $\sigma$ & $m$ & $\sigma$ & $m$ & $\sigma$ \\
\hline
\csname @@input\endcsname test4_2d.dat
\hline\hline
\end{tabular} }
\vspace*{0.1cm} 
\centerline{\tablesize \begin{tabular}{r|rr|rr|rr}
\hline\hline
\multicolumn{1}{c}{}&
\multicolumn{2}{|c}{$\|d_h(\phi_\mathrm{lin}\|$}&
\multicolumn{2}{|c}{$\|d_h(\phi_\mathrm{mgfas})\|$}&
\multicolumn{2}{|c}{$\|d_h(\phi_\mathrm{mudfas})\|$}\\
grid size $(k)$ & $m$ & $\sigma$ & $m$ & $\sigma$ & $m$ & $\sigma$ \\
\hline
\csname @@input\endcsname test5_2d.dat
\hline\hline
\end{tabular} }
\caption{Mean value and standard deviation
of the defect of the 2D potential as a function of the grid
size (and thus the number of grids $k$) for the 
long wave with \protect\input{partest4_2d.dat} (\emph{upper} table)
short wave with \protect\input{partest5_2d.dat} (\emph{lower} table)} 
\label{table:test45_2d}
\end{table}

\begin{table}
\centerline{\tablesize \begin{tabular}{r|rr|rr|rr}
\hline\hline
\multicolumn{1}{c}{}&
\multicolumn{2}{|c}{$\|d_h(\phi_\mathrm{lin})\|$}&
\multicolumn{2}{|c}{$\|d_h(\phi_\mathrm{mgfas})\|$}&
\multicolumn{2}{|c}{$\|d_h(\phi_\mathrm{mudfas})\|$}\\
grid size $(k)$ & $m$ & $\sigma$ & $m$ & $\sigma$ & $m$ & $\sigma$ \\
\hline
\csname @@input\endcsname test4_3d.dat
\hline\hline
\end{tabular} }
\vspace*{0.1cm}
\centerline{\tablesize \begin{tabular}{r|rr|rr|rr}
\hline\hline
\multicolumn{1}{c}{}&
\multicolumn{2}{|c}{$\|d_h(\phi_\mathrm{lin}\|$}&
\multicolumn{2}{|c}{$\|d_h(\phi_\mathrm{mgfas})\|$}&
\multicolumn{2}{|c}{$\|d_h(\phi_\mathrm{mudfas})\|$}\\
grid size $(k)$ & $m$ & $\sigma$ & $m$ & $\sigma$ & $m$ & $\sigma$ \\
\hline
\csname @@input\endcsname test5_3d.dat
\hline\hline
\end{tabular} }
\caption{Mean value and standard deviation
of the defect of the 3D potential as a function of the grid
size (and thus the number of grids $k$) for 
the long wave with \protect\input{partest4_3d.dat} (\emph{upper} table)
the short wave with \protect\input{partest5_3d.dat} (\emph{lower} table) }
\label{table:test45_3d}
\end{table}

Table~\ref{table:test45_2d} and \ref{table:test45_3d} present the mean and
standard deviation of the defect of the potential over $\Omega$ in 2D and 3D
respectively.  It is clearly seen that the more grids, the
higher accuracy the solver \mudfas achieves.

\subsection{Beam}

We have performed a numerical experiment with a density field occurring when
an ion beam is injected in a background plasma.
The beam is along the $y$-axis and the density takes the form
\begin{align}
\left\{\begin{array}{lll}
\rho(\vec{r})&=1+n_b&,  \qquad r_\perp<=d \mbox{ and }  r_\|<=H\\
\rho(\vec{r})&=1&, \qquad\mbox{otherwise}
\end{array}\right.
\end{align}
where $d$ and $H$ are the radius and length of the beam and $n_b$ its density.

The density of the beam and the electron temperature are 
\input{partest6_2d.dat} and the radius is $1/6$ of the width 
and the length is $1/2$ of the depth of the domain $\Omega$.

\begin{figure}
\centerline{\includegraphics[width=\columnwidth]{test6_2_2d}}
\centerline{\includegraphics[width=\columnwidth]{test6_2_3d}}
\caption{Experiment with a beam: \emph{upper} row, the defect of the 
2D potential, \emph{lower} row, the defect of a cut into the $xy$-plane of
the 3D potential}
\label{fig:6_2_23d}
\end{figure}

\begin{table}
\centerline{\tablesize \begin{tabular}{r|rr|rr}
\hline\hline
\multicolumn{1}{c}{}&
\multicolumn{2}{|c}{$\|d_h(\phi_\mathrm{mgfas})\|$}&
\multicolumn{2}{|c}{$\|d_h(\phi_\mathrm{mudfas})\|$}\\
grid size $(k)$ & $m$ & $\sigma$ & $m$ & $\sigma$ \\
\hline
\csname @@input\endcsname test6_2d.dat
\hline\hline
\end{tabular} }
\vspace*{0.1cm}
\centerline{\tablesize \begin{tabular}{r|rr|rr}
\hline\hline
\multicolumn{1}{c}{}&
\multicolumn{2}{|c}{$\|d_h(\phi_\mathrm{mgfas})\|$}&
\multicolumn{2}{|c}{$\|d_h(\phi_\mathrm{mudfas})\|$}\\
grid size $(k)$ & $m$ & $\sigma$ & $m$ & $\sigma$ \\
\hline
\csname @@input\endcsname test6_3d.dat
\hline\hline
\end{tabular} }
\caption{Mean value and standard deviation of the defect for the 2D potential 
(\emph{upper} table) and 3D potential (\emph{lower} table)
for the beam}
\label{table:test6_23d}
\end{table}

Figure~\ref{fig:6_2_23d} shows the defect of the potential in both 2D and 3D.
The defect is calculated for the grid size giving the largest number of
grids.
Table~\ref{table:test6_23d} presents the results as a function of the grid
size, i.e.\ the number of grids. For such a density field, the accuracy is
doubled by using more than or just 3 grids.

\subsection{Noise influence}

The accuracy of the defect will decrease with increasing noise. Suppose that
the density field $\rho$ of the beam is superimposed with a Gaussian
distributed noise with different value of standard deviation $\sigma_g$. 
This is just the situation when doing PIC simulation.

\begin{table} 
\centerline{\tablesize \begin{tabular}{l|rr|rr}
\hline\hline
\multicolumn{1}{c}{}& 
\multicolumn{2}{|c}{$\|d_h(\phi_\mathrm{mgfas})\|$}&
\multicolumn{2}{|c}{$\|d_h(\phi_\mathrm{mudfas})\|$}\\
$\sigma_g$ & $m$ & $\sigma$ & $m$ & $\sigma$ \\ 
\hline
\csname @@input\endcsname test7_2d.dat
\hline\hline
\end{tabular} 
\hspace*{0.1cm}
\begin{tabular}{l|rr|rr}
\hline\hline
\multicolumn{1}{c}{}&
\multicolumn{2}{|c}{$\|d_h(\phi_\mathrm{mgfas})\|$}&
\multicolumn{2}{|c}{$\|d_h(\phi_\mathrm{mudfas})\|$}\\
$\sigma_g$ & $m$ & $\sigma$ & $m$ & $\sigma$ \\
\hline
\csname @@input\endcsname test7_3d.dat
\hline\hline
\end{tabular} }
\caption{Mean and standard deviation of the defect for the 2D potential
(\emph{left} table) and 3D potential (\emph{right} table)
for different value of the noise standard deviation}
\label{table:test7_23d}
\end{table}

\subsection{Pre-/post-smoothing influence}

The number of pre- and post-smoothing is a parameter that can be used in
order to improve the accuracy of the solver. Here is a numerical experiment
that checks all the combinations of the number of pre- and post-smoothing
from 1 to 4.

\begin{table}
\centerline{\tablesize \begin{tabular}{l|rr}
\hline\hline
\multicolumn{1}{c}{}&
\multicolumn{2}{|c}{$\|d_h(\phi_\mathrm{mudfas})\|$}\\
(iprer, ipost) & $m$ & $\sigma$ \\
\hline
\csname @@input\endcsname test8_2d.dat
\hline\hline
\end{tabular} 
\hspace*{0.1cm}
\begin{tabular}{l|rr}
\hline\hline
\multicolumn{1}{c}{}&
\multicolumn{2}{|c}{$\|d_h(\phi_\mathrm{mudfas})\|$}\\
(iprer, ipost) & $m$ & $\sigma$ \\
\hline
\csname @@input\endcsname test8_3d.dat
\hline\hline
\end{tabular} }
\caption{Mean and standard deviation of the defect for the 2D potential
(\emph{left} table) and 3D potential (\emph{right} table) for different
combination of number of pre- and post-smmothing }
\label{table:test8_23d}
\end{table}

\section{Discussion}

We have performed a series of numerical experiment in order to test our
nonlinear Poisson solver \mudfas.
\mudfas is definitely more accurate than \mgfas, depending on the test
performed, varying from 4 times up to more 100 times more accurate.

It turns out that using a three-grid algorithm instead of a two-grid one
increases the accuracy performances and increasing even more the number of
grids can be useful to achieve better accuracy especially for low frequency
spatial structures.

The way \mudfas is implemented makes it very easy to adapt the
discretisation to other types of elliptic operator as for example Laplacian
in cylindrical or spherical coordinates.

\bibliography{abbrevs,research,books}

%\clearpage
\section*{Appendix: correspondence table  mesh size/grid number}

Following is a table giving the number of grids for grid size from 21 to
359. In each row, the grid size are classified from the one giving the
largest number to grid to the least.

\begin{table}[h]
\centerline{\tablesize \csname @@input\endcsname appendix.dat}
\caption{Number of grid points $n$ and the corresponding number of grids $k$
as well as the number of points on the coarsest grid $p$ }
\label{table:appendix}
\end{table}

\begin{table}[h]
\centerline{\tablesize \csname @@input\endcsname appendix2.dat}
\caption{Number of grid points $n$ and the corresponding number of grids $k$
as well as the number of points on the coarsest grid $p$ }
\label{table:appendix2}
\end{table}

\begin{table}[h]
\centerline{\tablesize \csname @@input\endcsname appendix3.dat}
\caption{Number of grid points $n$ and the corresponding number of grids $k$
as well as the number of points on the coarsest grid $p$ }
\label{table:appendix3}
\end{table}

\begin{table}[h]
\centerline{\tablesize \csname @@input\endcsname appendix4.dat}
\caption{Number of grid points $n$ and the corresponding number of grids $k$
as well as the number of points on the coarsest grid $p$ }
\label{table:appendix4}
\end{table}

\begin{table}[h]
\centerline{\tablesize \csname @@input\endcsname appendix5.dat}
\caption{Number of grid points $n$ and the corresponding number of grids $k$
as well as the number of points on the coarsest grid $p$ }
\label{table:appendix5}
\end{table}

\begin{table}[h]
\centerline{\tablesize \csname @@input\endcsname appendix6.dat}
\caption{Number of grid points $n$ and the corresponding number of grids $k$
as well as the number of points on the coarsest grid $p$ }
\label{table:appendix6}
\end{table}

\begin{table}[h]
\centerline{\tablesize \csname @@input\endcsname appendix7.dat}
\caption{Number of grid points $n$ and the corresponding number of grids $k$
as well as the number of points on the coarsest grid $p$ }
\label{table:appendix7}
\end{table}

\begin{table}[h]
\centerline{\tablesize \csname @@input\endcsname appendix8.dat}
\caption{Number of grid points $n$ and the corresponding number of grids $k$
as well as the number of points on the coarsest grid $p$ }
\label{table:appendix8}
\end{table}

\end{document}

